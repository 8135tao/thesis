%%%%%%%%%%%%%%%%%%%%%%%%%%%%%%%%%%%%%%%%%%%%%%%%%%%%%%%%%%%%%%%%%%%%%%%%%%%%%%%%
% app_glossary.tex: Glossary Appendix:
%%%%%%%%%%%%%%%%%%%%%%%%%%%%%%%%%%%%%%%%%%%%%%%%%%%%%%%%%%%%%%%%%%%%%%%%%%%%%%%%
\chapter{Passage through fold}
\label{app_p2}
%%%%%%%%%%%%%%%%%%%%%%%%%%%%%%%%%%%%%%%%%%%%%%%%%%%%%%%%%%%%%%%%%%%%%%%%%%%%%%%%


%%%%%%%%%%%%%%%%%%%%%%%%%%%%%%%%%%%%%%%%%%%%%%%%%%%%%%%%%%%%%%%%%%%%%%%%%%%%%%%%
We first show how to extend the asymptotic expansion \eqref{ric_asy} to a more general family of solutions.
\section{A family of solutions to the Riccati equation}

\begin{Proof}[\textbf{ of Proposition \ref{para_ric}}] To get the dependence from $u_0$ to $\Omega_\infty$, we first add the equation $\frac{d}{ds}s=1$ to \eqref{ric} to get an autonomous $2-$dimensional system in the $(s,u)$ plane. Consider a small neighborhood $I$ containing $\bar{u}_0$ on the vertical $u$-axis, then $u_R(s; u_0)$ is the trajectory that starts at $u_0 \in I$. The map $P_1 : I \to \mathbb{R}$ defined by $P_1(p) = u(2; p)$ is smooth in $p$, as the blow up time for $\bar{u}_R(s;\bar{u}_0)$ is $\Omega_0 >2$. Moreover, the image $P_1(I)$ is a finite interval on the vertical line $s=2$ containing $\bar{u}_R(2;u_0)$ bounded away from $0$, since the trajectory $u_R(s;\bar{u}_0)$ crosses the horizontal axis around $s=1$ and the vector field goes upwards in the first quadrant of the $(s,u)$-plane.

Denote $\tilde{u}_0:= P_1(u_0)$ for brevity (technically, the interval $P_1(I)$ is a small section of the line $s=2$, with a little abuse of notation, we identify $\tilde{u}_0$ with the second coordinate of the point $P_1(u_0)$). Again we make a change of variable in \eqref{ric} by setting $z(s) = 1/u(s)$. Then $z$ satisfies 
\[
\frac{d}{ds}z(s) = -z^2s -1.
\]
Let $J = \{ 1/\tilde{u}_0 \mid  \tilde{u}_0 \in P_1(I)\}$ and $z(s; 1/\tilde{u}_0)$ is the trajectory which starts at $1/\tilde{u}_0$. We claim that $z(s; 1/\bar{u}_0)$ reaches $0$ at a finite time $\Omega_\infty = \Omega_\infty(1/\bar{u}_0)$. To see this, first notice there is no equilibrium for the two dimensional system $\frac{d}{ds}s=1, \frac{d}{ds}z=-z^2s-1$. Then, on the boundary $s=2$, the vector field takes the form $(1,-2z^2-1)$, which makes any trajectory starting at a point on $J$ moving down towards the right. Moreover, the vector field $(1,-sz^2-1)$ always pointing down in the first quadrant of the $(s,z)$ plane, so trajectories cannot go upwards. Lastly, the vector field crosses the horizontal axis non-tangentially, it identically equals $(1,-1)$ throughout the line $z=0$, hence, any trajectory which starts at a point on $J$ will cross $z=0$ in finite time at a unique point $\Omega_\infty = \Omega_\infty(1/\tilde{u}_0)$. The dependence of $\Omega_\infty$ on $1/\tilde{u}_0$ is smooth by the smooth dependence on initial conditions.

 We now define another map $P_2 : J \to \mathbb{R}$ by $P_2(1/\tilde{u}_0) = \Omega_\infty(1/\tilde{u}_0)$, we get a smooth map $P: I \to \mathbb{R}$ by the composition
 \[
 P =P_2 \circ f \circ P_1,
 \] 
 where $f(z) = 1/z$ is the inversion map. Since each of the map in the composition is smooth, $P: u_0 \mapsto \Omega_\infty = \Omega_\infty(u_0)$ is smooth as well.

To get the asymptotic expansion, we set $\xi = \Omega_\infty-s$, then $\tilde{z}(\xi)=z(\Omega_\infty-\xi)$ solves
\[
\frac{d}{d\xi} \tilde{z} = \tilde{z}^2(\Omega_\infty-\xi)+1,
\]
and $\tilde{z}(0) = 0$.

Hence we can assume the expansion for $\tilde{z}$ near $\xi=0$ is of the form
\[
\tilde{z} = \xi + z_2\xi^2+z_3\xi^3 + \rmO(\xi^4),
\]
for some constant $z_2,z_3$. Differentiating this expansion, use the equation $\tilde{z}$ solves and comparing coefficients, we find that $z_2 = 0, z_3 = \Omega_\infty/3$.  Changing back from $\tilde{z}(\xi)$ to $z=z(s)$ with $s = \Omega_\infty-\xi$ and recall $z(s) = 1/u(s)$, we find that $u_R(s;u_0)$ has expansion \eqref{ric_exp} with remainder $r$ satisfies \eqref{ric_reminder}.


\end{Proof}


\section{Uniform invertibility of boundary value problems }
Next we show the main perturbation lemma used to prove the invertibility of the linearized operators at the ansatz.
\begin{Lemma}\label{pert_inv}
Consider the following boundary value problems 
\begin{subequations}
\label{lin_bv}
\begin{align}
       &\dot{u}(\sigma) = a(\sigma) u + f(\sigma), \hspace{0.2in} u(L)= u_L,         \label{eqn_pos_line}  \\
              &\dot{u}(\sigma) = b(\sigma) u + g(\sigma), \hspace{0.2in} u(-M)= u_M,         \label{eqn_neg_line}
\end{align}
\end{subequations}

where \eqref{eqn_pos_line} is posed on $\sigma \in (\sigma_0,L)$ with $L>\sigma_0$ and \eqref{eqn_neg_line} is posed on $\sigma \in (-M,\sigma_0)$ with $M>\sigma_0$. 

Assume $a(\sigma), b(\sigma)$ are continuous functions such that 
\begin{subequations}
\label{ode_asy}
\begin{align}
       &a(\sigma) \to a_+>0, \hspace{0.2in} \sigma \to \infty      ,  \label{ode_asy_a}  \\
       &b(\sigma) \to b_- < 0, \hspace{0.2in} \sigma \to -\infty,         \label{ode_asy_b}
\end{align}
\end{subequations}
then \eqref{lin_bv} possess unique solutions $u_a,u_b$ which satisfies
\begin{subequations}
\label{ode_est}
\begin{align}
       &|u_a|_\infty \le C_a(u_L+|f|_\infty),  \label{ode_est_a}  \\
       &|u_b|_\infty \le C_b(u_m+|g|_\infty),         \label{ode_est_b}
\end{align}
\end{subequations}
for some constants $C_a, C_b$ independent of $L$ and $M$.
\end{Lemma}

\begin{Proof}
We only prove the estimate \eqref{eqn_pos_line} since the other case is similar. Also, without loss of generality, we assume that $\sigma_0 = 0$.

To begin with, consider the asymptotic equation 
\begin{equation}\label{asy_eq}
\dot{u} = a_+ u + f(\sigma),\hspace{ 0.5in } u(L) = u_L.
\end{equation}
posed on $\sigma \in [0, L]$.
Then the estimate \eqref{ode_est_a} holds for \eqref{asy_eq} since in this case we have
\begin{align*}
u(\sigma) &= e^{a_+(\sigma-L)}u_L + \int_L^t e^{a_+(\sigma-s)} f(s)ds \\ 
&\le 2|u_L| + \left|\int_L^t e^{a_+(\sigma-s)}ds \right| |f|_\infty\\ 
&\le  2|u_L|+\frac{1}{a_+} \left|e^{t-L}-1\right||f|\infty\\
& \le 2(|u_L|+|f|_\infty ).
\end{align*}

Next, give $\eta>0$ small enough and independent of $L$, there exist $\sigma_* \le L$ such that $|a(\sigma)- a_+|< \eta$ for all $\sigma>\sigma*$. It is important to note here that one can choose $\sigma_*$ independent of $L$ as long as $L$ is large enough. A Neumann series argument shows that in this case the operator 
\[ u \mapsto
 \left(\frac{d}{dt}u-a(t)u, u(L)\right)
\] is a $\eta-$perturbation of the asymptotic operator
\[ u \mapsto
 \left(\frac{d}{dt}u-a_+u, u(L)\right),
\]
as a densely defined operator on $\mathcal{C}^1(\sigma_*,L) \subset \mathcal{C}(\sigma_*,L)$, so similarly from step I, we conclude that 
\[
\sup_{\sigma \in (\sigma_*,L)} |u(\sigma)| \le C(|u|_L + |f|_\infty)
\]
for some constant $C$ independent of $L$.

Finally, for $\sigma \in (0,\sigma_*)$, the solution is given by the following formula
\[
u(\sigma) = \exp\left(\int^{\sigma}_{\sigma_*} a(\tau)d\tau\right) u(\sigma_*) + \int_{\sigma_*}^{\sigma} \exp\left(-\int_{\sigma}^{s}a(\tau)d\tau\right)f(s)ds 
\]
since $\sigma_*< \infty$ and does not depend on $L$, there exist a constant $C_1$ independent of $L$ so that 
\[
\max\left\{ \left|\exp\left(\int^{\sigma}_{\sigma_*} a(\tau)d\tau\right)\right|, \left| \int_{\sigma_*}^{\sigma} \exp\left(-\int_{\sigma}^{s}a(\tau)d\tau\right)\right| \right\} \le C_1.
\]
Moreover, the value $u(\sigma_*)$ satisfies
\[
u(\sigma_*) \le \sup_{\sigma \in (\sigma_*,L)} |u(\sigma)| \le C_2(u_L + |f|_\infty)
\]
for some constant $C_2$ independent of $L$ from the conclusion in step 2.
Therefore on $[0,\sigma_*]$ the solution satisfies
\[
\sup_{\sigma \in [0,\sigma_*]}|u(\sigma)| \le C_1C_2(u_L+|f|_\infty) +C_1|f|_\infty \le C(u_L+|f|_\infty)
\]
where the constant $C$ does not depend on $L$. Therefore we conclude that
\[
\sup_{\sigma \in [0,L]} = |u|_\infty \le C(u_L+|f|_\infty)
\]
which is \eqref{ode_est_a}.
%\cite{plantdisp}
\nocite{*}
\end{Proof}
%\subsubsection{Matching at \texorpdfstring{$\sigma^*$}{sigma^*} }

%%%%%%%%%%%%%%%%%%%%%%%%%%%%%%%%%%%%%%%%%%%%%%%%%%%%%%%%%%%%%%%%%%%%%%%%%%%%%}}}

%%%%%%%%%%%%%%%%%%%%%%%%%%%%%%%%%%%%%%%%%%%%%%%%%%%%%%%%%%%%%%%%%%%%%%%%%%%%%}}}
